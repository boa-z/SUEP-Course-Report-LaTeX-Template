\documentclass{course-thesis/theme-2614084}
\usepackage{hyperref}
\usepackage{bookmark}
\usepackage{minted}

\usepackage{hyperref}

% Build: xelatex --shell-escape course-thesis/theme-2614084-example.tex

% =============================================
% Part 0 信息
% =============================================

\mathsetup{
  % 学生姓名
  student-name = {某同学},
  % 学号
  student-id = {2021xxxx},
  % 院系
  department = {电子与信息工程学院},
  % 专业
  experiment = {实验4 CMOS与非门版图设计},
  % 专业年级
  major = {集成电路设计与集成系统},
  % 日期
  % date = {\today},
}

\begin{document}

% =============================================
% Part 1  封面
% =============================================

\makecover

% =============================================
% Part 2 主文档
% =============================================

\section{实验目的}

\begin{enumerate}
  \item 熟悉 virtuoso editing 设计窗口及操作,熟悉 LSW 窗口
  \item 理解设计库、技术库
  \item 了解 SMIC0.18um 工艺规则
  \item 认识 DRC 设计规则检查,排除错误
\end{enumerate}

\section{实验环境}

\begin{enumerate}
  \item 硬件: PC 机、服务器
  \item 环境:Unix 操作系统、Cadence Virtuoso Editing 版图设计软件
\end{enumerate}

\section{实验内容与步骤}

% (注:按照内容,有截图和说明)

\subsection{实验内容}

\begin{enumerate}
  \item 按照设计规则进行CMOS与非门的版图设计。
  \item 进行CMOS与非门的DRC、LVS检查,生成GDS 文件。
\end{enumerate}

\subsection{实验步骤}

\section{实验总结和感悟}

\end{document}
