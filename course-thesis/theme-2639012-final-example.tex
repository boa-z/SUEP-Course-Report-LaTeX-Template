% 修改此处以选择使用的模板
\documentclass{course-thesis/theme-2639012-final}

\usepackage{hyperref}
\usepackage{bookmark}
\usepackage{minted}

% Build: xelatex --shell-escape course-thesis/theme-2639012-final-example

% =============================================
% Part 0 信息
% =============================================

\mathsetup{
  % 学院
  institution = {电子与信息工程学院},
  % 专业
  discipline = {集成电路设计与集成系统},
  % 课程编号
  course-number = {2639012.01},
  % 学生姓名
  student-name = {某同学},
  % 学号
  student-id = {2021****},
  % 班级
  student-class = {集成21391},
  % 指导老师
  teacher = {杨程},
  % 日期——年
  year = {2024},
  % 日期——月
  month = {4},
  % 日期——日
  day= {30},
}

\begin{document}

% =============================================
% Part 1  封面
% =============================================

\makecover

% =============================================
% Part 2 主文档
% =============================================

\section{课程设计目的}

\section{课程设计任务一(15.1.4)}

\subsection{实验目的与内容}

\subsection{实验步骤}

% 2.1 对照 /projet_soc/TP/TP0 和/projet_soc/TP/TP0_CORRECTION 里面的如下个文件,并将 /projet_soc/TP/TP0 中的这6个文件中与 /projet_soc/TP/TP0_CORRECTION 中的这个6个文件不同的地方改成跟/projet_soc/TP/TP0_CORRECTION中的这6个文件相同的形式:
% /projet_soc/TP/TP0/HW/里面的个文件:
% 2.2 更新top.cpp中的部分代码
% 2.3 segmentation.h
% 2.4 platform_desc
% /projet_soc/TP/TP0/SW/mjpeg_seq/headers/里面的个文件:
% 1. fetch.h 改目录:
% movie=fopen("/fd/home/vlsi/projet_soc/TP/TP0/SW/mjpeg_seq/images/ice_age256x144_144.mjpeg","r")
% /projet_soc/TP/TP0/SW/mjpeg_seq/ldscripts/里面的个文件:
% 1. mips
% /projet_soc/TP/TP0/SW/mjpeg_seq/sources/里面的个文件:
% 1. dispatch.c
% 编译仿真:268页(3)编译仿真步骤1-5.

\subsection{实验过程}

\section{课程设计任务二(15.1.5)}

\section{课程设计总结}

% 请删除\todo部分,并开始编写你的总结
\todo[inline,color=blue!20]{需包含与书中1-14章(主要是第4章)相关的内容之间的联系,根据自己的理解来写即可。)}

\end{document}
